\documentclass[12pt,landscape]{report}
\usepackage{cmap}
\usepackage[T1,T2A]{fontenc}
\usepackage[utf8]{inputenc}
\usepackage[english,russian]{babel}
\usepackage{geometry}
\usepackage[usenames,dvipsnames,svgnames,table]{xcolor}
\usepackage{array}

\geometry{left=1cm}
\geometry{right=1cm}
\geometry{top=2cm}
\geometry{bottom=2cm}

\input defines.tex
\input colors.tex
% \input colors_laser.tex           % changes all colors to white

\begin{document}
\pagenumbering{gobble}
\centering

% | => !{\vrule width 2pt}
% \hline => \noalign{\hrule height 2pt}

\begin{tabular}{@{}|c!{\vrule width 1.25pt}c|c|c|c|c|c|@{}}
\hline
~ &
\textbf{Понедельник} &
\textbf{Вторник} &
\textbf{Среда} &
\textbf{Четверг} &
\textbf{Пятница} &
\textbf{Суббота}
\\ \noalign{\hrule height 1.25pt}
%
% 1 пара
\ddd{1}{~9:00}{10:35} &

\ppps{Обычная пара} &
\ppps{Заполнитель} &
\ppps{Мигающая пара} &
\ppps{Две мигающих пары} &
~ &
~

\\ \hline
%
% 2 пара
\ddd{2}{10:50}{12:25} &

\ppps{\texttt{\textbackslash ppp\\{\small\{Что\}\{Где\}\{Кто ведет\}}}} &
\ppps{\texttt{\textbackslash ppps\\{\small\{Что\}}}} &
\ppps{\texttt{\textbackslash pppf\\{\small\{Что\}\{Где\}\{Кто ведет\}\{Неч|Ч\}}}} &
\ppps{\texttt{\textbackslash ppppp\\{\small\{Что1\}\{Где1\}\{Кто ведет1\}\\\{Что2\}\{Где2\}\{Кто ведет2\}}}} &
~ &
~


\\ \hline\hline
%
% 3 пара
\ddd{3}{13:30}{15:05} &

\ppp{Что}{Где}{Кто ведет} &
\ppps{Что} &
\pppf{Что}{Где}{Кто ведет}{Неч|Ч} &
\ppppp{Что1}{Где1}{Кто ведет1}{Что2}{Где2}{Кто ведет2} &
~ &
~

\\ \hline
%
% 4 пара
\ddd{4}{15:20}{16:55} &

~ &
~ &
~ &
~ &
~ &
~

\\ \hline
%
% 5 пара
\ddd{5}{17:05}{18:40} &

\blue \ppps{\texttt{\textbackslash blue}} &
\green \ppps{\texttt{\textbackslash green}} &
\orange \ppps{\texttt{\textbackslash orange}} &
\magenta \ppps{\texttt{\textbackslash magenta}} &
\yellow \ppps{\texttt{\textbackslash yellow}} &
\violet \ppps{\texttt{\textbackslash violet}}

\\ \hline

\end{tabular}

\vfill\hfill{\color{gray!50}\scriptsize https://github.com/xtotdam/timetable}
\end{document}
