\documentclass[12pt,landscape]{report}
\usepackage{cmap}
\usepackage[T1,T2A]{fontenc}
\usepackage[utf8]{inputenc}
\usepackage[english,russian]{babel}
\usepackage{geometry}
\usepackage[usenames,dvipsnames,svgnames,table]{xcolor}
\usepackage{array}

\geometry{left=1cm}
\geometry{right=1cm}
\geometry{top=2cm}
\geometry{bottom=2cm}

\input defines.tex
\input colors.tex
% \input colors_laser.tex           % changes all colors to white

\begin{document}
\pagenumbering{gobble}
\centering

% | => !{\vrule width 2pt}
% \hline => \noalign{\hrule height 2pt}

\begin{tabular}{@{}|c!{\vrule width 1.25pt}c|c|c|c|c|c|@{}}
\hline
~ &
\textbf{Понедельник} &
\textbf{Вторник} &
\textbf{Среда} &
\textbf{Четверг} &
\textbf{Пятница} &
\textbf{Суббота}
\\ \noalign{\hrule height 1.25pt}
%
% 1 пара
\ddd{1}{~9:00}{10:35} &

~ &          % пн
\pppmil &                                               % вт
\blue \ppp{Фотонные кристаллы}{4-30}{Манцызов Б.И.} &                  % ср
{\color{gray} \ppp{Матметоды в физике}{5-68}{Силаев П.К.}   } &                                               % чт
~ &                                                     % пт
\pppsprak                                               % сб

\\ \hline
%
% 2 пара
\ddd{2}{10:50}{12:25} &

{\color{gray} \ppp{?}{?}{Грановский А.Б.}   } &                                               % пн
\pppmil &                                               % вт
\blue \ppp{Что-то о лю\-ми\-нес\-цен\-ци\-и}{4-28}{Салецкий А.М.} &        % ср
~ &                                                     % чт
~ &                                                     % пт
\pppsprak                                               % сб

\\ \hline\hline
%
% 3 пара
\ddd{3}{13:30}{15:05} &

\pppnir &                                               % пн
\pppmil &                                               % вт
\blue \ppp{Что-то с биологией}{4-30}{Брандт Н.Б.} &                                                     % ср
\orange \ppp{Английский язык}{5-45}{Коваленко И.Ю.} &                   % чт
\blue \ppp{ДМП}{4-30}{Русаков В.С.} &                   % пт
{\color{gray} \ppp{Эконофизика}{4-30}{Тишин А.М.}   }          % сб

\\ \hline
%
% 4 пара
\ddd{4}{15:20}{16:55} &

\green \ppp{Экономика}{ЦФА}{Катихин О.В.} &             % пн
~ &                                                     % вт
\yellow \pppmfk &                                       % ср
~ &                                                     % чт
~ &                                                     % пт
~                                                       % сб

\\ \hline
%
% 5 пара
\ddd{5}{17:05}{18:40} &

~ &                                                     % пн
~ &                                                     % вт
\yellow \pppmfk &                                       % ср
~ &                                                     % чт
~ &                                                     % пт
~                                                       % сб

\\ \hline

\end{tabular}

\vfill\hfill{\color{gray!50}\scriptsize https://github.com/xtotdam/timetable}
\end{document}
