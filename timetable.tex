% !TeX TS-program = lualatex

\documentclass[12pt]{report}

\usepackage{fontspec}

% from https://github.com/AndreyAkinshin/Russian-Phd-LaTeX-Dissertation-Template/
\setmonofont{CMU Typewriter Text}
\newfontfamily\cyrillicfonttt{CMU Typewriter Text}
\defaultfontfeatures{Ligatures=TeX}
\setmainfont{CMU Serif}
\newfontfamily\cyrillicfont{CMU Serif}
\setsansfont{CMU Sans Serif}
\newfontfamily\cyrillicfontsf{CMU Sans Serif}

\usepackage{polyglossia}
\setmainlanguage[babelshorthands=true]{russian}
\setotherlanguage{english}

\usepackage{geometry}
\geometry{
	a4paper,
	landscape,
	left=0.5cm,
	right=0.5cm,
	top=1.5cm,
	bottom=1cm
}

\usepackage{fontawesome5}
\usepackage[usenames,dvipsnames,svgnames]{xcolor}
\usepackage{tikz}
\usetikzlibrary{math, calc, patterns}

\begin{document}

\pagenumbering{gobble}
\centering

\tikzmath{
	% Часы и минуты -> минуты
	function cft(\hour, \min){return \hour * 60 + \min;};
	% Разница между двумя временами в минутах
	function difftime(\h1, \m1, \h2, \m2) {return ((\h2 - \h1) * 60 + \m2 - \m1) / 60.;};
	% Часы и минуты -> часы с дробной частью
	function t2f(\h, \m) {return \h + \m / 60.;};
}

\tikzset{nosep/.style={inner sep=0,outer sep=0}}

% Ширина и высота колонок и ячеек
\newcommand{\xunit}{4.5cm}
\newcommand{\yunit}{2cm}

% Первый и последний час 
\newcommand{\starthour}{8}
\newcommand{\finishhour}{15}

% Первый и последний день (1 -- 6)
\newcommand{\startdays}{1}
\newcommand{\finishdays}{5}

% Уровень нижней границы полосы с названиями дней
\newcommand{\dayslevel}{8.5}

\begin{tikzpicture}[x=\xunit,y=-\yunit]

% Названия дней - заголовок таблицы
\begin{scope}[every node/.style={draw,above right,minimum width=\xunit,minimum height=1.7em,align=center,nosep}]
	\node at (1, \dayslevel) {Понедельник};
	\node at (2, \dayslevel) {Вторник};
	\node at (3, \dayslevel) {Среда};
	\node at (4, \dayslevel) {Четверг};
	\node at (5, \dayslevel) {Пятница};
%	\node at (6, \dayslevel) {Суббота};
\end{scope}

% Левый столбец с часами
\foreach \x in {\starthour,...,\finishhour}
{
	\node[draw,below left,minimum height=\yunit,minimum width=2em,align=center] (hour-{\x}) at (1, \x) {\x}; 
}

% Ячейки слева с номерами пар и временем
\begin{scope}[every node/.style={draw,above left,minimum height=3em,minimum width={t2f(1, 35)*\yunit},align=center,nosep,
		xshift=-2.3em,rotate=90}]
	\node (p1) at (\startdays, {t2f( 9,  0)})	{1 пара\\ 9:00 --- 10:35};
	\node (p2) at (\startdays, {t2f(10, 50)}) 	{2 пара\\10:50 --- 12:25};
	\node (p3) at (\startdays, {t2f(13, 30)}) 	{3 пара\\13:30 --- 15:05};
%	\node (p4) at (\startdays, {t2f(15, 20)}) 	{4 пара\\15:20 --- 16:55};
%	\node (p5) at (\startdays, {t2f(17,  5)}) 	{5 пара\\17:05 --- 18:40};
\end{scope}

% Заштрихованные полосы, соотвествующие парам
\begin{scope}[ultra thick,black!20]
	\foreach \x in {1,...,3}{
		\fill[pattern=north east lines,nearly transparent] ( p\x.0 -| \startdays,0 ) rectangle ( p\x.180 -| \finishdays+1,0 );
	}
\end{scope}

\tikzmath{\fhp = \finishhour + 1; \sdp = \startdays + 1; \fdp = \finishdays + 1;}
% Горизонтальные линии - границы часов
\foreach \x in {\starthour,...,\fhp} {\draw[black,dotted] (\startdays, \x) -- (\fdp, \x);}
% Вертикальные линии - границы дней
\foreach \x in {\sdp,...,\fdp}{ \draw (\x, \dayslevel) -- (\x, \fhp); }

\tikzset{
	% одна пара по высоте (по умолчанию)
	single/.style={minimum height={t2f(1, 35)*\yunit}},
	% смена в практикуме = 3 часа 25 минут
	prac/.style={minimum height={t2f(3, 25)*\yunit}},
	% ранняя смена чуть длиннее
	prac2/.style={minimum height={t2f(3, 30)*\yunit}},
	% дистанционная пара
	dist/.style={dashed},
	% ``выключенная'' пара
	greyedout/.style={text=black!50},
}

% 8:30 - ранняя смена практикума
\coordinate (prac1) at (1, {t2f(8, 30)});

% дни недели
\coordinate (mon) at (1,0); \coordinate (tue) at (2,0); \coordinate (wed) at (3,0);
\coordinate (thu) at (4,0); \coordinate (fri) at (5,0); \coordinate (sat) at (6,0);

% пары
\coordinate (p1) at (0,{t2f( 9, 0)}); \coordinate (p2) at (0,{t2f(10,50)});
\coordinate (p3) at (0,{t2f(13,30)}); \coordinate (p4) at (0,{t2f(15,20)});
\coordinate (p5) at (0,{t2f(17, 5)});

% Собственно, пары
\begin{scope}[every node/.style={draw,single,below right,fill=white,minimum width=\xunit,text width=\xunit,align=center,nosep}]

\node[prac2,fill=pink!40] at (mon |- prac1) {
	\textbf{Практикум}\\[1em]Оптика\\Задачи 408, 410\\[1em]Гр. 207, 209
};

\node[prac2,fill=pink!40] at (thu |- prac1) {
	\textbf{Практикум}\\[1em]Оптика\\Задачи 147, 142, 152, 169, 413\\[1em]Гр. 204, 211
};

\node[dist,fill=yellow!30] at (wed |- p1) {
	\textbf{Семинар}\\Введение в квантовую физику\\Группа 212
};

\end{scope}


% Жирные линии, отделяющие заголовки колонок от таблицы
\draw[thick] (\startdays,\dayslevel) -- (\finishdays+1,\dayslevel);
\draw[thick] (\startdays,\starthour) -- (\startdays,\finishhour+1);

% github
\node[below left,color=black!30,font=\scriptsize] at (\fdp,\fhp) {\faGithub\texttt{/xtotdam/timetable}};
\end{tikzpicture}

\end{document}