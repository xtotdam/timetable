% !TeX TS-program = lualatex

\documentclass[12pt]{report}

\usepackage{fontspec}

% from https://github.com/AndreyAkinshin/Russian-Phd-LaTeX-Dissertation-Template/
\setmonofont{CMU Typewriter Text}
\newfontfamily\cyrillicfonttt{CMU Typewriter Text}
\defaultfontfeatures{Ligatures=TeX}
\setmainfont{CMU Serif}
\newfontfamily\cyrillicfont{CMU Serif}
\setsansfont{CMU Sans Serif}
\newfontfamily\cyrillicfontsf{CMU Sans Serif}

\usepackage{polyglossia}
\setmainlanguage[babelshorthands=true]{russian}
\setotherlanguage{english}

\usepackage{geometry}
\geometry{
	a4paper,
	landscape,
	left=1cm,
	right=1cm,
	top=2cm,
	bottom=2cm
}

\usepackage[usenames,dvipsnames,svgnames]{xcolor}
\usepackage{tikz}
\usetikzlibrary{math, calc, patterns}

\begin{document}

\pagenumbering{gobble}
\centering

\tikzmath{
	function cft(\hour, \min){return \hour * 60 + \min;};
	function difftime(\h1, \m1, \h2, \m2) {return ((\h2 - \h1) * 60 + \m2 - \m1) / 60.;};
	function t2f(\h, \m) {return \h + \m / 60.;};
}

\tikzset{
	nosep/.style={inner sep=0,outer sep=0},
}

\newcommand{\xunit}{4.5cm}
\newcommand{\yunit}{2cm}

\newcommand{\starthour}{8}
\newcommand{\finishhour}{15}

\newcommand{\startdays}{1}
\newcommand{\finishdays}{5}

\newcommand{\dayslevel}{8.5}

\begin{tikzpicture}[x=\xunit,y=-\yunit]

\begin{scope}[every node/.style={draw,above right,minimum width=\xunit,minimum height=1.7em,align=center,nosep}]
	\node at (1, \dayslevel) {Понедельник};
	\node at (2, \dayslevel) {Вторник};
	\node at (3, \dayslevel) {Среда};
	\node at (4, \dayslevel) {Четверг};
	\node at (5, \dayslevel) {Пятница};
%	\node at (6, \dayslevel) {Суббота};
\end{scope}

\foreach \x in {\starthour,...,\finishhour}
{
	\node[draw,below left,minimum height=\yunit,minimum width=2em,align=center] (hour-{\x}) at (1, \x) {\x}; 
}

\begin{scope}[every node/.style={draw,above left,minimum height=3em,minimum width=(1+35/60)*\yunit,align=center,nosep,
		xshift=-2.3em,rotate=90}]
	\node (p1) at (\startdays,9) 			{1 пара\\9:00 --- 10:35};
	\node (p2) at (\startdays,10+50/60) 	{2 пара\\10:50 --- 12:25};
	\node (p3) at (\startdays,13.5) 		{3 пара\\13:30 --- 15:05};
%	\node (p4) at (\startdays,15+20/60) 	{4 пара\\15:20 --- 16:55};
%	\node (p5) at (\startdays,17+5/60) 		{5 пара\\17:05 --- 18:40};
\end{scope}

\begin{scope}[ultra thick,black!20]
	\foreach \x in {1,...,3}{
		\fill[pattern=north east lines,nearly transparent] ( p\x.0 -| \startdays,0 ) rectangle ( p\x.180 -| \finishdays+1,0 );
	}
\end{scope}

\foreach \x in {\starthour,...,\finishhour} \draw[black,dotted] (\startdays,\x) -- (\finishdays+1,\x);
\draw[black!10] (\startdays,\finishhour+1) -- (\finishdays+1,\finishhour+1);
\foreach \x in {2,...,6}{ \draw (\x,\dayslevel) -- (\x,\finishhour+1); }

\begin{scope}[every node/.style={draw,below right,minimum width=\xunit,text width=\xunit,align=center,nosep}]

\node[minimum height=3.5*\yunit,fill=pink!40] at (1,8.5) {\textbf{Практикум}\\[1em]Оптика\\Задачи 408, 410\\[1em]Гр. 207, 209};
\node[minimum height=3.5*\yunit,fill=pink!40] at (4,8.5) {\textbf{Практикум}\\[1em]Оптика\\Задачи 147, 142, 152, 169, 413\\[1em]Гр. 204, 211};

\node[dashed,minimum height={t2f(1,35)*\yunit},fill=yellow!30] at (3,9) {\textbf{Семинар}\\Введение в квантовую физику\\Группа 212};

\end{scope}

\draw[thick] (\startdays,\dayslevel) -- (\finishdays+1,\dayslevel);
\draw[thick] (\startdays,\starthour) -- (\startdays,\finishhour+1);
\end{tikzpicture}

\end{document}